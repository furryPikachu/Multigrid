\begin{defn}
  A system of ordinary differential equations(ODEs)
  of dimension $N$ is a set of differential equations of the form
  \begin{equation}
    \label{eq:ode}
    \mathbf{u}'(t) = \mathbf{f}(\mathbf{u}(t),t),
  \end{equation}
  where $t$ is time,
  $\mathbf{u}\in \mathbb{R}^N$ is the evolutionary varible,
  and the RHS function has the signature
  $\mathbf{f}:\mathbb{R}^N\times (0,+\infty)\rightarrow \mathbf{R}^N$.
  In particular, (\ref{eq:ode}) is an ODE for $N=1$.
\end{defn}

\begin{defn}
  A system of ODEs is \emph{linear} if its RHS function can be
  expressed as
  $\mathbf{f}(\mathbf{u},t)=\alpha(t)\mathbf{u}+\beta(t)$,
  and \emph{nonliner} otherewise.
\end{defn}

\begin{exm}
  For the simple pendulum shown above,
  the moment of interial and the torque are
  \begin{displaymath}
    I=m\ell^2,\tau=-mg\ell\sin \theta,
  \end{displaymath}
  and the equation of motion can be derived from Newton's second
  law $\tau=I\theta''(t)$ as
  \begin{equation}\label{eq:motion}
    \theta''(t)=-\frac{g}{\ell}\sin\theta,
  \end{equation}
  which admits a unique solution if we impose two initial conditions
  \begin{displaymath}
    \theta(0)=\theta_0,\theta'(0)=\omega_0.
  \end{displaymath}
  Alternatively, (\ref{eq:motion}) can be derived by the consideration
  that the total energy remains a constant with respect to time.
  \begin{displaymath}
    \begin{aligned}
      L=\frac{1}{2}m(\ell\theta')^2+mg\ell(1-\cos\theta);\\
      \frac{\dif L}{\dif t}= 0\Rightarrow m\ell^2\theta'\theta''
      +mg\ell \theta'\sin \theta=0.
    \end{aligned}
  \end{displaymath}
  The ODE (\ref{eq:ode}) is second-order, nonlinear, and autonomous;
  it can be reduced to a first-order system as follows,
  \begin{displaymath}
    \omega =\theta',\mathbf{u}=
    \begin{pmatrix}
      \theta\\
      \omega
    \end{pmatrix}
    \Rightarrow m\ell^2\theta'\theta''+mg\ell\theta'\sin\theta =0.
  \end{displaymath}
\end{exm}

\begin{defn}\label{defn:IVP}
  Given $T>0$, $\mathbf{f}:\mathbb{R}^N\times [0,T]\rightarrow
  \mathbb{R}^N$, and $\mathbf{u}_0\in \mathbb{R}^N,$
  the \emph{initial value problem} (IVP) is to find
  $\mathbf{u}(t)\in{\cal C}^1$ such that
  \begin{equation}
    \label{eq:ivp}
    \begin{cases}
      \mathbf{u}(0)=\mathbf{u}_0,\\
      \mathbf{u}'(t)=\mathbf{f}(\mathbf{u}(t),t),\forall t\in [0,T].
    \end{cases}
  \end{equation}
\end{defn}

\begin{defn}
  The IVP in Definiton \ref{defn:IVP} is \emph{well-posed} if
  \begin{enumerate}
  \item [(i)] it admits a unique solution for any fixed $t>0$,
  \item [(ii)] $\exists c>0,\hat{\epsilon}>0$ s.t.
    $\forall \epsilon< \hat{\epsilon}$, the perturbed IVP
    \begin{equation}
      \label{eq:perturbedIVP}
      \mathbf{v}'=\mathbf{f}(\mathbf{v},t)+\mathbf{\delta}(t),
      \quad,\mathbf{v}(0)=\mathbf{u}_0+\mathbf{\epsilon}_0
    \end{equation}
satisfies
\begin{equation}
  \forall t\in [0,T],
  \begin{cases}
    \Vert \mathbf{\epsilon}_0\Vert  < \epsilon\\
    \Vert \mathbf{\delta}(t)\Vert 
  \end{cases}
  \Rightarrow \Vert \mathbf{u}(t)-\mathbf{v}(t)\Vert\leq c\epsilon.
\end{equation}
  \end{enumerate}
\end{defn}

\subsection{Lipschitz continuity}

\label{sec:lipschitz-continuity}

\begin{defn}
  A function $\mathbf{f}:\mathbb{R}^N\times [0,+\infty)\rightarrow
  \mathbb{R}^N$ is \emph{Lipschitz continuous} in its first variable
  over some domain
  \begin{equation}
    \label{eq:domain}
    {\cal D}=\{(\mathbf{u},t):\Vert\mathbf{u}-\mathbf{u}_0\Vert
    \leq a,t\in [0,T]\}
  \end{equation}
if
\begin{equation}
  \exists L\geq 0 \textmd{ s.t. } \forall (\mathbf{u},t),
  (\mathbf{v},t)\in {\cal D},\Vert \mathbf{f}(\mathbf{u},t)-\mathbf{f}
  (\mathbf{v},t)\Vert\leq L\Vert \mathbf{u}-\mathbf{v}\Vert.
\end{equation}
\end{defn}

\begin{exm}
  If $\mathbf{f}(\mathbf{u},t)=\mathbf{f}(t)$, then $L=0$.
\end{exm}

\begin{exm}
  If $\mathbf{f}\notin {\cal C}^0,$ then $\mathbf{f}$ is not Lipschitz.
\end{exm}

\begin{defn}
  A subset of $S\subset\mathbb{R}^n$ is \emph{star-shaped} with
  respect to a point $p\in S$ if for each $x\in S$ the line segment
  from $p$ to $x$ lies in $S$.
\end{defn}

\begin{thm}\label{thm:star-shaped}
  Let $S\subset \mathbb{R}^n$ be star-shaped with respect to
  $p=(p_1,p_2,\ldots,p_n)\in S.$
  For a continuously differentiable function $f: S\rightarrow
  \mathbb{R}$,
  there exist continuously differentiable functions $g_1(\mathbf{x}),
  g_2(\mathbf{x}),\ldots,g_n(\mathbf{x})$ such that
  \begin{equation}
    f(\mathbf{x})=f(p)+\sum_{i=1}^{n}(x_i-p_i)g_i(\mathbf{x}),
    g_i(p)=\frac{\partial f}{\partial x_i}(p).
  \end{equation}
\end{thm}

\begin{prop}
  If $\mathbf{f}(\mathbf{u},t)$ is continuously differentiable on some
  compact convex set ${\cal D}\subseteq \mathbb{R}^{N+1}$,
  then \textbf{f} is Lipschitz on ${\cal D}$ with
  \begin{displaymath}
    L=\max_{i,j}\left\vert \frac{\partial f_i}{\partial u_j}\right\vert.
  \end{displaymath}
\end{prop}

\begin{lem}
  Let $(M,\rho)$ denote a complete metric space and $\phi:M\rightarrow
  M $ a contractive mapping in the sense that
  \begin{equation}
    \exists c\in [0,1], \textmd{ s.t.}
\forall \eta,\zeta \in M, \rho(\phi(\eta),\phi(\zeta))\leq c\rho(\eta,\zeta).
  \end{equation}
\end{lem}

\begin{thm}[Fundamental theorem of ODEs]
If $\mathbf{f}(\mathbf{u}(t),t)$ is Lipschitz continuous in
$\mathbf{u}$ and continuous in $t$ over some region ${\cal D}=\{(\mathbf{u},t):\Vert\mathbf{u}-\mathbf{u}_0\Vert\leq a,t\in [0,T]\}$,
then there is a unique solution to the IVP problem as in Defintion
\ref{defn:IVP} at least up to time $T^*=\min (T,\frac{a}{S})$
where $S=\max_{(\mathbf{u},t)\in{\cal D}}\Vert \mathbf{f}(\mathbf{u},t)\Vert$
\end{thm}

\begin{defn}
  If $\mathbf{f}(\mathbf{u},t)$ is Lipschitz in $\mathbf{u}$ and
  continuous in $t$ on ${\cal
    D}=\{(\mathbf{u},t):\Vert\mathbf{u}-\mathbf{u}_0\Vert\leq a,t\in
  [0,T]\}$, then the IVP in Defintion \ref{defn:IVP} is well-posed
  for all initial data.
\end{defn}

\begin{exm}
  Consider $N=1,u'(t)=\sqrt{u(t)},u(0)=0.$
  \begin{displaymath}
    \lim_{u\rightarrow 0} f'(u)=\lim_{u\rightarrow 0}
    \frac{1}{2\sqrt{u}}=+\infty.
  \end{displaymath}
  Hence $f(u)$ is not Lipschitz near $u=0$.
  However, $u(t)\equiv 0$ and $u(t)=\frac{1}{4}t^2$ are both
  solutions.
  Hence the Lipschitz condition is not necessary for existence.
\end{exm}

\begin{exm}
  Consider the IVP $u'(t)=u^2,u_0=\eta>0.$
  The slope $f'(u)=2u \rightarrow +\infty$ as $u\rightarrow \infty$.
  So there is no unique solution on $[0,+\infty),$ but there might
  exist
  $T^*$ such that unique solutions are guaranteed on $[0,T^*]$.

  In fact, $u(t)=\frac{1}{\eta^{-1}-t}$ is a solution,
  but blows up at $t=1/\eta.$
  According to Theorem 
\end{exm}


\subsection{Some basic numerical methods}

\label{sec:some-basic-numerical}
\begin{ntn}
  In the following, we shall use $k$ to denote the time step,
  and thus $t_n=nk$.
\end{ntn}

To numerically solve the IVP (\ref{eq:ivp}) we are given
initial data $\mathbf{U}^0=\mathbf{u}_0,$
and want to compute approximations $\mathbf{U}^1,\mathbf{U^2},\ldots,$
such that
\begin{displaymath}
  \mathbf{U}^n\approx \mathbf{u}(t_n).
\end{displaymath}

\begin{defn}
  The \emph{forward Euler's method} solves the IVP (\ref{eq:ivp})
  by
  \begin{equation}
    \mathbf{U}^{n+1} = \mathbf{U}^n+k\mathbf{f}(\mathbf{U}^n,t_n)
  \end{equation}
  which is based on replacing $\mathbf{u}'(t_{n})$ with the forward
  difference $(\mathbf{U}^{n+1}-\mathbf{U}^n)/k$ and
  $\mathbf{u}(t_{n})$
  with $\mathbf{U}^{n}$ in $\mathbf{f}(\mathbf{u},t)$.
\end{defn}

\begin{defn}
 The \emph{backward Euler's method} solves the IVP (\ref{eq:ivp})
  by
  \begin{equation}
    \mathbf{U}^{n+1} = \mathbf{U}^n+k\mathbf{f}(\mathbf{U}^{n+1},t_{n+1})
  \end{equation}
  which is based on replacing $\mathbf{u}'(t_{n+1})$ with the backward
  difference $(\mathbf{U}^{n+1}-\mathbf{U}^n)/k$ and
  $\mathbf{u}(t_{n+1})$
  with $\mathbf{U}^{n+1}$ in $\mathbf{f}(\mathbf{u},t)$.
\end{defn}

\begin{defn}
  The \emph{trapezoidal method} is
  \begin{equation}
    \label{eq:trapezoidal}
    \mathbf{U}^{n+1}=\mathbf{U}^n+\frac{k}{2}(\mathbf{f}(\mathbf{U}^{n},t_n),\mathbf{f}(\mathbf{U}^{n+1},t_{n+1})).
  \end{equation}
\end{defn}

\begin{defn}
  The \emph{midpoint (or leapfrog) method} is
  \begin{equation}
    \label{eq:midpoint}
    \mathbf{U}^{n+1}=\mathbf{U}^{n-1}+2k\mathbf{f}(\mathbf{U}^{n},t_n).
  \end{equation}
\end{defn}

\subsection{Accuracy and convergence}
\label{sec:accuracy-convergence}

\begin{defn}
  The \emph{local truncation error}(LTE) is the error caused by
  replacing
  continuous derivatives with finite difference formulas.
\end{defn}

\begin{exm}
  For the leapfrog method, the local truncation error is
  \begin{displaymath}
    \begin{aligned}
      \mathbf{\tau}^n&=\frac{\mathbf{u}(t_{n+1})-\mathbf{u}(t_{n})}{2k}
      -\mathbf{f}(\mathbf{u}(t_n),t_n)\\
      &= \left [ \mathbf{u}'(t_n)+\frac{1}{6}k^2\mathbf{u}'''(t_n)+
        O(k^4)\right ]-\mathbf{u}'(t_n)\\
      &=\frac{1}{6}k^2\mathbf{u}'''(t_n)+O(k^4).
    \end{aligned}
  \end{displaymath}
\end{exm}

\begin{defn}
  For a numerical method of the form
  \begin{displaymath}
    \mathbf{U}^{n+1}=\phi( \mathbf{U}^{n+1}, \mathbf{U}^{n},
    \ldots, \mathbf{U}^{n-m}),
  \end{displaymath}
the \emph{one-step error} is defined by
\begin{equation}
  \label{eq:one-step}
  {\cal L}^n:=\mathbf{u}(t_{n+1})-\phi(\mathbf{u}(t_{n+1}),
  \mathbf{u}(t_{n}),\ldots,\mathbf{u}(t_{n-m})).
\end{equation}
\end{defn}

\begin{exm}
  For the leapfrog method, the one-step error is
  \begin{displaymath}
    \begin{aligned}
      {\cal
        L}^n&=\mathbf{u}(t_{n+1})-\mathbf{u}(t_{n-1})-2k\mathbf{f}(\mathbf{u}(t_n),t_n)\\
      &=\frac{1}{3}k^3\mathbf{u}'''(t_n)+O(k^5)\\
      &=2k\mathbf{\tau}^n
    \end{aligned}
  \end{displaymath}
\end{exm}

\begin{defn}
  The \emph{solution error} of a numerical method for solving the IVP
  in Definiton \ref{defn:IVP} is
  \begin{equation}
    \label{eq:solutionError}
    \mathbf{E}^N:=\mathbf{U}^{T/k}-\mathbf{u}(T);\quad
    \mathbf{E}^n=\mathbf{U}^n-\mathbf{u}(t_n).
  \end{equation}
\end{defn}

\begin{defn}\label{defn:convergent}
  A numerical method is convergent for a family of IVPs if the
  application of it to any IVP with $\mathbf{f}(\mathbf{u},t)$
  Lipschitz continuous in $\mathbf{u}$ and continuous in $t$ yields
  \begin{equation}
    \label{eq:convergence}
    \lim_{\substack{k\rightarrow 0\\Nk=T}}=\mathbf{u}(T)
  \end{equation} 
for every fixed $T> 0$.
\end{defn}

\subsection{Analysis of Euler's method}

\label{sec:analys-eulers-meth}

\subsubsection{Linear IVPs}

\label{sec:linear-ivps}
In this section, we consider the convergence of Euler's method
for solving linear IVPs of the form
\begin{equation}
  \label{eq:linearIVPs}
  \begin{cases}
    \mathbf{u}'(t)=\lambda\mathbf{u}(t)+\mathbf{g}(t),\\
    \mathbf{u}(0)=\mathbf{u}_0,
  \end{cases}
\end{equation}
where $\lambda$ is either a scalar or a diagonal matrix.

\begin{lem}
  Solution errors and the local truncation error of Euler's method
  applied to the linear IVP (\ref{eq:linearIVPs}) satisfy
  \begin{equation}
    \label{eq:solutionErrorLinear}
    \mathbf{E}^{n+1}=(1+k\lambda)\mathbf{E}^n-k\mathbf{\tau}^n.
  \end{equation}
\end{lem}

\begin{lem}
  The relation between the solution error $\mathbf{E}^n$ and the local
  trucation error $\tau^n$ of Euler's method applied to the linear
  IVP (\ref{eq:linearIVPs}) is
  \begin{equation}
    \label{eq:relationSoluLocal}
    \mathbf{E}^n=(1=k\lambda)^n\mathbf{E}^0-k\sum_{m=1}^n
    (1+k\lambda)^{n-m}\tau^{m-1}.
  \end{equation}
\end{lem}

\begin{thm}
  Euler's method is convergent for solving the linear IVP (\ref{eq:linearIVPs}).
\end{thm}

\subsubsection{Nonlinear IVPs}

\label{sec:nonlinear-ivps}
In this section, we consider the convergence of Euler's method
for solving nonlinear IVPs of the form
\begin{equation}\label{eq:nonlinarIVPs}
  \mathbf{u}'(t)=\mathbf{f}(\mathbf{u}(t),t),
\end{equation}
with $\mathbf{f}(\mathbf{u},t)$ Lipschitz continuous in $\mathbf{u}$
(with Lipschitz constant $L$)
and continuous in $t$.

\begin{lem}\label{lem:nonlinearError}
  Under the above assumption, we have
  \begin{equation}
    \label{eq:nonlinearError}
    \Vert \mathbf{E}^{n+1}\Vert \leq (1+kL)\Vert \mathbf{E}^{n}\Vert
    +k\Vert \tau^{n}\Vert.
  \end{equation}
\end{lem}

\begin{thm}
  Under the same assumption as in Lemma \ref{lem:nonlinearError},
  Euler's method is convergent.
\end{thm}

\subsubsection{Absolute stability}

\label{sec:absolute-stability}
\begin{exm}
  Consider the ODE
  \begin{displaymath}
    u'(t)=\lambda(u-\cos t)-\sin t,
  \end{displaymath}
  with $\lambda=-2100$ and $u(0)=1.$
  The exact solution is clearly
  \begin{displaymath}
    u(t)=\cos t.
  \end{displaymath}
  The following table shows the 
\begin{verbatim}
    k     |      E(T)  
-----------------------
2.00e-04  |   1.98e-08    
4.00e-04  |   3.96e-08    
8.00e-04  |   7.92e-08    
9.50e-04  |   5.77e-08    
9.76e-04  |   1.75e+36    
1.00e-03  |   1.45e+76   
\end{verbatim}
\end{exm}

%%% Local Variables:

%%% mode: latex
%%% TeX-master: "../notesNumericalSolution"
%%% End:
